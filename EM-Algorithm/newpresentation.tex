\documentclass[11pt]{beamer}
\usetheme{Warsaw}
\usepackage[utf8]{inputenc}
\usepackage{amsmath}
\usepackage{amsfonts}
\usepackage{amssymb}
\author{Ritwik Sahani/ Aayush Goyal}
\title{The EM Algorithm}
%\setbeamercovered{transparent} 
%\setbeamertemplate{navigation symbols}{} 
%\logo{} 
%\institute{} 
%\date{} 
%\subject{} 
\institute{IIT Hyderabad}
\begin{document}

\begin{frame}
\titlepage{}
\end{frame}
\begin{frame}




\bigskip
\textsf{Suppose we have some data points x_{1}......x_{n}}
\begin{figure}
\includegraphics[scale=.3]{pic1.png}
\end{figure}
\textsf{and we also tell you that these points belong to two different probabilistic distributions, say, two gaussians such that we know which point belogs to which particular distribution.}\linebreak\linebreak
\textbf{Can we find the parameters $\mu_{b}, \sigma^{2}$ for the two gaussians?}\linebreak\linebreak

\textbf{Indeed, we can.}\linebreak\linebreak
\textsf{$\mu = \sum_{i=1}^{n_{b}}\frac{x_{i}}{n_{b}}$}\linebreak
\textsf{$\sigma^{2} = \sum_{i=1}^{n_{b}}\frac{(x_{i} - \mu_{b})^{2}}{n_{b}}$\linebreak
and similarly for the other distribution.}

\end{frame}



\begin{frame}



\textsf{Now, suppose data points are there but we don't know the as to which gaussian  a particular point belongs.}
\begin{figure}
\includegraphics[scale=.3]{pic3.png}
\end{figure}
\textsf{But what we do know is that these come from two gaussians, whose parameters $\mu, \sigma^{2}$ is known}\linebreak\linebreak
\textbf{Can we, for each point, decide which of the two gaussians it is more likely to belong to?}\linebreak
\textsf{Yes, this can be done by claculating a posteriori prob.}\linebreak\linebreak
\textsf{$P(b|x_{i}) =\frac{ P(x_{i}|b)P(b)}{P(x_{i}|b)P(b) + P(x_{i}|a)P(a)}$}\linebreak
\textsf{where $P(x_{i}|b)$ can be obtained from the gaussian $N(\mu_{b}, \sigma_{b}^{2})$}
\end{frame}


\begin{frame}
\begin{figure}
\includegraphics[scale=.3]{pic2.png}
\end{figure}
\textbf{But what happens, if we neither knew the source gaussian, nor their parameters?}\linebreak\linebreak
\textsf{Then we use the EM algorithm method, an iterative method, to find the same.}
\end{frame}


\begin{frame}
\textbf{Step-1: Expectation}\linebreak
\linebreak
\textsf{Creates a function for the expectation of the log-likelihood evaluated using the
current estimate for the parameters.}\linebreak
$\star$
\textbf{ Basically we will assume some initial parameters and assign each point with its a- posteriori probability}
\linebreak
\linebreak
\linebreak

\textbf{Step-2: Maximization}
\linebreak

\textsf{Which computes parameters maximizing the expected log-likelihood found on the
E step. These parameter-estimates are then used to determine the distribution of
the latent variables in the next E step.}\linebreak
$\star$
\textbf{ Basically, we will again calculate the parameters for the next E -step }

\end{frame}






\begin{frame}
\textbf{Two-Component Mixture Model}\linebreak
\textsf{We consider a simple mixture model for density estimation,
and the corresponding EM algorithm for carrying out maximum likelihood
estimation.}



\begin{figure}

\includegraphics[scale=0.17]{fig1.png}
\label{Histogram of data}
\caption{Mixture example. (Left panel:) Histogram of data. (Right panel:)
Maximum likelihood fit of Gaussian densities (solid red) and responsibility (dotted green) of the left component density for observation y, as a function of y.}






\end{figure}

\end{frame}


\begin{frame}
\begin{figure}
\includegraphics[scale=0.15]{fig3.jpg}
\end{figure}
\textsf{Due to Bimodalty of data Gaussian distribution would be inappropriate. So we
will model Y as a mixture of two distinct distributions.\linebreak\linebreak
$Y_{1} \equiv N(\mu_{1},  \sigma_{1}^2) $\linebreak
$Y_{2} \equiv N(\mu_{2},  \sigma_{2}^2) $\linebreak
$Y &= (1-\Delta)*Y_{1} + \Delta * Y_{2} $\linebreak
where $\Delta \in \{0, 1\}$ with $Pr(\Delta &= 1) = \pi $\linebreak\linebreak
This generative representation is explicit: generate a $\Delta \in \{0, 1\} $ with
probability $\pi$, and then depending on the outcome, deliver either Y_{1}  or  Y_{2} .
Let $\phi$ $\theta$ (x) denote the normal density with parameters $\theta = (\mu, \sigma_{2} )$. Then the
density of Y is\linebreak
$gY(y) = (1- \pi)\phi \theta_{1} (y)  + \pi\phi \theta_{2} (y).$}

\end{frame}


\begin{frame}
\textsf{Now suppose we wish to fit this model to the data in Figure 8.5 by maximum likelihood.The parameters are\linebreak
$\theta = (\pi, \theta_{1} , \theta_{2}) = (\pi, \mu_{1} , \sigma_{12} , \mu_{2} , \sigma_{22})$\linebreak
The log-likelihood based on the N training cases is}\linebreak\linebreak
$l(\theta; Z) = \sum_{i=1}^{N}  log[(1- \pi)\phi \theta_{1} (y_{i}) + \\pi\phi \theta_{2} (y_{i} )]$\linebreak\linebreak
\textsf{Direct maximization of $l(\theta; Z)$ is quite difficult numerically, because of
the sum of terms inside the logarithm. There is, however, a simpler approach. We consider unobserved latent variables $\theta_{i}$ taking values 0 or 1 as
in (8.36): if $\Delta_{i} = 1 $then $Y_{1}$ comes from model 2, otherwise it comes from
model 1.\linebreak\linebreak
$l_{0}(\theta;Z,\Delta) = \sum_{i=1}^{N}[(1-\Delta_{i})log\phi_{\theta_{1}}(y_{i}) + \Delta_{i}log\phi_{\theta_{2}}(y_{i})] + \sum_{i=1}^{N}[(1-\Delta_{i})log(1-\pi)(y_{i}) + \Delta_{i}log\pi]$}
\end{frame}


\begin{frame}

\textsf{\linebreak\linebreak
Now we apply the E-M Algorithm for Two-component Gaussian mixture}
\begin{figure}
\includegraphics[scale=.215]{fig10.jpg}
\end{figure}
\end{frame}

\begin{frame}{•}
Note that the actual maximization of the likelihood occurs when we put a
spike of infinite height at any one data point, that is, $\hat{\mu}_{1} = y_{i}$ for some
i and $\hat{\sigma}_{1}^2 = 0$. This gives infinite likelihood, but is not a useful solution.
Hence we are actually looking for a good local maximum of the likelihood,
one for which $\hat{\sigma}_{1}^2 , \hat{\sigma}_{2}^2 > 0.$ To further complicate matters, there can be
more than one local maximum having $\hat{\sigma}_{1}^2 , \hat{\sigma}_{2}^2 > 0.$. In our example, we
ran the EM algorithm with a number of different initial guesses for the
parameters, all having $\hat{\sigma}_{k}^2 > 0.5$, and chose the run that gave us the highest
maximized likelihood

\end{frame}
\begin{frame}
\begin{figure}
  
  \includegraphics[width=200]{fig4.jpg}
  
  \caption{Selected iterations for E-M algorithm}
  \label{fig:boat1}
\end{figure}
\textsf{The final maximum likelihood estimates are $\hat{\mu}_{1} = 4.62$                 $\hat{\sigma}_{1}^2 = 0.87$, $\hat{\mu}_{2} = 1.06 $ ,$\hat{\sigma}_{2}^2 = 0.77$, $\hat{\pi}$ = 0.546.}
\end{frame}

\begin{frame}

\begin{figure}
  \includegraphics[width=200]{fig5.jpg}
  
  \caption{observed data log-likelihood as a function of the iteration number}
  \label{fig:boat1}

\end{figure}

\end{frame}
\begin{frame}
\textsf{EM Algorithm in general}
\begin{figure}
  \includegraphics[width=300]{fig11.png}
  
 

\end{figure}

\end{frame}


\begin{frame}
\textsf{References}
\bigskip
\linebreak
\textsf{1) The Elements of Statistical Learning - Trevor Hastie, Robert Tibshirani, Jerome Friedman}
\medskip
\linebreak 

\textsf{2) https://en.wikipedia.org/}
\medskip
\linebreak 

\textsf{3) Victor Lavrenko - Youtube}
\end{frame}


\end{document}
